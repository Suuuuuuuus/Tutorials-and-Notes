\documentclass[UTF8]{book}
\usepackage{ctex}
\usepackage{amsmath}
\usepackage{bm}
\usepackage{makeidx}
\usepackage{enumitem}
\usepackage{rotating} 
\usepackage{yhmath}
\usepackage{textcomp,booktabs}
\usepackage[usenames,dvipsnames]{color}
\usepackage{colortbl}
\usepackage{makecell}
\usepackage{gensymb}
\usepackage{cancel}
\usepackage{graphicx}
\usepackage{pifont}
\usepackage[all,pdf]{xy}
\usepackage{exscale}
\usepackage{blindtext}
\usepackage{hyperref}
\hypersetup{
colorlinks=true,
linkcolor=black
}
\usepackage{nameref}
\usepackage{relsize}
\usepackage{titlesec}
\usepackage{ifthen}
\usepackage{array}
\usepackage[flushleft]{threeparttable}
\usepackage{diagbox}
\usepackage{mathtools}
\usepackage{amsfonts}
\usepackage{bbm}
\usepackage{ulem}
\usepackage{xcolor}
\usepackage{color}
\usepackage{mathptmx}
\usepackage{amssymb}
\usepackage{geometry}
\geometry{a4paper, left=2.54cm,right=2.54cm,top=2.54cm,bottom=2.54cm}
%\geometry{b5paper, left=1.6cm,right=2cm,top=2cm,bottom=2cm}
\usepackage{mathrsfs}
\usepackage{tikz,tkz-euclide}
\usepackage{tikz}
\usetikzlibrary{patterns}
\usetikzlibrary{shapes,arrows}
\usepackage{esvect}
\usepackage{fancyhdr}
\pagestyle{fancy}
\fancyhead{} % clear all fields 
\cfoot{}
\fancyhead[LE,RO]{\thepage} 
\renewcommand{\headrulewidth}{0pt}
\renewcommand{\footrulewidth}{0pt}
\linespread{1.4}
\date{}
\graphicspath{ {../Graphs} }
\usepackage{listings}
\usepackage{color}

\definecolor{dkgreen}{rgb}{0,0.6,0}
\definecolor{gray}{rgb}{0.5,0.5,0.5}
\definecolor{mauve}{rgb}{0.58,0,0.82}

\lstset{
  basicstyle=\ttfamily,
  columns=fullflexible,
  frame=single,
  breaklines=true,
  backgroundcolor=\color{gray!20!white}
}
\definecolor{codegray}{gray}{0.8}
\newcommand{\code}[1]{\colorbox{codegray}{\texttt{#1}}}
%通用:
\newcounter{mylabelcounter}
\makeatletter
\newcommand{\labeltext}[2]{%
#1\refstepcounter{mylabelcounter}%
\immediate\write\@auxout{%
  \string\newlabel{#2}{{1}{\thepage}{{\unexpanded{#1}}}{mylabelcounter.\number\value{mylabelcounter}}{}}%
}%
}
%Xsum
\DeclareFontFamily{U} {cmex}{}
\DeclareFontShape{U}{cmex}{m}{n}{
  <-6> cmex5
  <6-7> cmex6
  <7-8> cmex7
  <8-9> cmex8
  <9-10> cmex9
  <10-12> cmex10
  <12-> cmex12}{}
\DeclareSymbolFont{Xcmex} {U} {cmex}{m}{n}
\DeclareMathSymbol{\Xdsum}{\mathop}{Xcmex}{88}
\DeclareMathSymbol{\Xtsum}{\mathop}{Xcmex}{80}
\DeclareMathOperator*{\Xsum}{\mathchoice{\Xdsum}{\Xtsum}{\Xtsum}{\Xtsum}}
%Xsum
%choice
\newcommand{\fourch}[4]{%~\hfill(\qquad)\\
\begin{tabular}{*{4}{@{}p{0.25\textwidth}}}(A)~#1 & (B)~#2 & (C)~#3 & (D)~#4\end{tabular}}
\newcommand{\twoch}[4]{%~\hfill(\qquad)\\
\begin{tabular}{*{2}{@{}p{0.5\textwidth}}}(A)~#1 & (B)~#2\end{tabular}\\\begin{tabular}{*{2}{@{}p{0.5\textwidth}}}(C)~#3 & (D)~#4\end{tabular}}
\newcommand{\onech}[4]{%~\hfill(\qquad)\\
(A)~#1 \\ (B)~#2 \\ (C)~#3 \\ (D)~#4}
 
\newlength\widthcha
\newlength\widthchb
\newlength\widthchc
\newlength\widthchd
\newlength\widthch
\newlength\tabmaxwidth
\setlength\tabmaxwidth{1\textwidth}
\newlength\fourthtabwidth
\setlength\fourthtabwidth{0.25\textwidth}
\newlength\halftabwidth
\setlength\halftabwidth{0.5\textwidth}
\newcommand{\choice}[4]{\settowidth\widthcha{AM.#1}\setlength{\widthch}{\widthcha}
    \settowidth\widthchb{BM.#2}
    \ifthenelse{\widthch<\widthchb}{\setlength{\widthch}{\widthchb}}{}
    \settowidth\widthchb{CM.#3}
    \ifthenelse{\widthch<\widthchb}{\setlength{\widthch}{\widthchb}}{}
    \settowidth\widthchb{DM.#4}
    \ifthenelse{\widthch<\widthchb}{\setlength{\widthch}{\widthchb}}{}
    \ifthenelse{\widthch<\fourthtabwidth}{\fourch{#1}{#2}{#3}{#4}}
    {\ifthenelse{\widthch<\halftabwidth\and\widthch>\fourthtabwidth}{\twoch{#1}{#2}{#3}{#4}}
        {\onech{#1}{#2}{#3}{#4}}}}
%choice
% Default fixed font does not support bold face
\DeclareFixedFont{\ttb}{T1}{txtt}{bx}{n}{12} % for bold
\DeclareFixedFont{\ttm}{T1}{txtt}{m}{n}{12}  % for normal

% Custom colors
\usepackage{color}
\definecolor{deepblue}{rgb}{0,0,0.5}
\definecolor{deepred}{rgb}{0.6,0,0}
\definecolor{deepgreen}{rgb}{0,0.5,0}

\usepackage{listings}

% Python style for highlighting
\newcommand\pythonstyle{\lstset{
language=Python,
basicstyle=\ttm,
morekeywords={self},              % Add keywords here
keywordstyle=\ttb\color{deepblue},
emph={MyClass,__init__},          % Custom highlighting
emphstyle=\ttb\color{deepred},    % Custom highlighting style
stringstyle=\color{deepgreen},
frame=single,                         % Any extra options here
showstringspaces=false
}}


% Python environment
\lstnewenvironment{python}[1][]
{
\pythonstyle
\lstset{#1}
}
{}

% Python for external files
\newcommand\pythonexternal[2][]{{
\pythonstyle
\lstinputlisting[#1]{#2}}}

% Python for inline
\newcommand\pythoninline[1]{{\pythonstyle\lstinline!#1!}}
%%%%%
% Bash style for highlighting
\newcommand\bashstyle{\lstset{
language=bash,
basicstyle=\ttm\normalsize,
tabsize=4,
morekeywords={self, head, tail, uniq, sort, grep, cat, cut, echo, wc, cp, rm, mkdir, cd, nano, man, ls, history, bash, rmdir, find, plink, bcftools, bedtools, octopus},              % Add keywords here
keywordstyle=\color{deepblue}\normalsize\bfseries,
emph={MyClass,__init__},          % Custom highlighting
emphstyle=\ttb\color{deepred},    % Custom highlighting style
stringstyle=\color{deepgreen},
frame=single,                         % Any extra options here
showstringspaces=false
}}


% Python environment
\lstnewenvironment{bash}[1][]
{
\bashstyle
\lstset{#1}
}
{}

% Python for external files
\newcommand\bashexternal[2][]{{
\bashstyle
\lstinputlisting[#1]{#2}}}

% Python for inline
\newcommand\bashinline[1]{{\bashstyle\lstinline!#1!}}
%%%%%
\newcommand{\dollar}{\mbox{\textdollar}}
\newcommand{\dps}[1]{\ensuremath{\displaystyle{#1}}}
\newcommand\ffrac[2]{\ensuremath{\dfrac{\;#1\;}{\;#2\;}}}
\newcommand{\comma}{\, \; \;\mathclap{\text{,}}} %用于mathmode中的逗号
\newcommand{\semicolon}{\, \; \;\mathclap{\text{;}}} %用于mathmode中的分号
\newcommand{\pl}{\phantom{l}} %用来占位
\newcommand{\un}{\ding{172}}
\newcommand{\deux}{\ding{173}}
\newcommand{\trois}{\ding{174}}
\newcommand{\quatre}{\ding{175}}
\newcommand{\et}{&}
\newcommand{\f}{^2}
\newcommand{\xz}{(\qquad)}
\newcommand{\tk}{\underline{\qquad\qquad}}
%高等数学:
\renewcommand{\d}{\,\mathrm{d}}
\newcommand{\dt}{\,\mathrm{d}t}
\newcommand{\dr}{\,\mathrm{d}r}
\newcommand{\du}{\,\mathrm{d}u}
\newcommand{\dv}{\,\mathrm{d}v}
\newcommand{\dx}{\,\mathrm{d}x}
\newcommand{\dy}{\,\mathrm{d}y}
\newcommand{\dz}{\,\mathrm{d}z}
\newcommand{\df}{\,\mathrm{d}f}
\newcommand{\bigmid}{\, \bigg | \,} %用于集合中有分数的情况
\newcommand\matharr{\tikz[baseline=-0.4ex]\draw[-stealth] (0,0) -- + (3mm,0);} %用于下标中的右箭头
\newcommand\textarr{\; \tikz[baseline=-0.55ex]\draw[-stealth] (0,0) -- + (4mm,0);} %用于文本中的右箭头,注意用占位符调整前后间距
\newcommand{\limite}[2]{\ensuremath{\lim\limits_{#1\matharr #2}}} %#1趋向于#2
\newcommand{\dlimite}[4]{\ensuremath{\displaystyle{\lim_{\substack{ \phantom{l}#1\matharr #2\phantom{l} \\ #3\matharr #4}}}}} %#重极限:1趋向于#2,#3趋向于#4,phantom{l}用来占位
\newcommand{\neighbr}{\ensuremath{\mathring{U}(x_0\comma \delta)}} %去心邻域U
\newcommand{\neighbor}{\ensuremath{U(x_0\comma \delta)}} %邻域U
\newcommand{\tikzrm}[1]{
	\fill[white] #1 circle(1.5pt);
	\draw #1 circle(1.5pt);
}
\newcommand{\derivee}[4]{
	\ffrac{\,\mathrm{d}^{#1}#2}{\,\mathrm{d}#3^{#4}}
}
\newcommand{\intscript}[2]{\biggl.\biggr|_{\, #2}^{\, #1}} %求出原函数以后代入的积分上下限
\newcommand{\dint}[2]{\ensuremath{\displaystyle{\int_{#2}^{#1}}}}
\newcommand{\diint}[4]{\ensuremath{\displaystyle{\int_{#2}^{#1}\int_{#4}^{#3}}}}
\newcommand{\bint}{\mathlarger{\int}} %用于将幂次上的积分号放大
\newcommand{\exiint}{\ensuremath{\!\!\!}} %用于缩短累次积分中积分号的距离
\newcommand{\fxy}{\ensuremath{f(x\comma y)}}
\newcommand{\xoyo}{\ensuremath{(x_0\comma y_0)}}
\newcommand{\series}{\ensuremath{\dps{\Xsum_{n=1}^\infty}}} %级数
\newcommand{\serieso}{\ensuremath{\dps{\Xsum_{n=0}^\infty}}} %0开始的级数
%线性代数:
\newcommand{\pA}{\ensuremath{\pmb{A}}}
\newcommand{\pB}{\ensuremath{\pmb{B}}}
\newcommand{\pC}{\ensuremath{\pmb{C}}}
\newcommand{\pO}{\ensuremath{\pmb{O}}}
\newcommand{\pP}{\ensuremath{\pmb{P}}}
\newcommand{\pQ}{\ensuremath{\pmb{Q}}}
\newcommand{\pE}{\ensuremath{\pmb{E}}}
\newcommand{\px}{\ensuremath{\pmb{x}}}
\newcommand{\pX}{\ensuremath{\pmb{X}}}
\newcommand{\pR}{\ensuremath{\pmb{R}}}
\newcommand{\pZ}{\ensuremath{\pmb{Z}}}
\newcommand{\pal}{\ensuremath{\pmb{\alpha}}}
\newcommand{\pbe}{\ensuremath{\pmb{\beta}}}
\newcommand{\pxi}{\ensuremath{\pmb{\xi}}}
\newcommand{\pet}{\ensuremath{\pmb{\eta}}}
\renewcommand{\t}{\ensuremath{^\mathrm{T}}}
\newcommand\laarr{\qquad\tikz\draw[-stealth] (0,0) -- + (7mm,0);\qquad} %用于矩阵中的初等变换
\newcommand{\laarrt}[1]{\qquad\tikz\draw[-stealth] (0,0) -- (4mm,0) node[above]{#1}--+ (4mm,0);\qquad} %初等变换上带字
%概率:
\newcommand{\XY}{\ensuremath{(X\comma Y)}}
\newcommand{\Cov}{\ensuremath{\mathrm{Cov}}}
\newcommand{\cip}{\tikz[baseline=-0.55ex]\draw[-stealth] (0,0) -- (2mm,0) node[above]{$\;\;P$}--+ (4mm,0);\;} %依概率收敛
\newcommand{\seriesn}{\ensuremath{\dps{\Xsum_{i=1}^n}}} %1开始到n的连续求和
\begin{document}
\kaishu
\begin{center}
\Large{Git Software Carpentry Notes}
\end{center}
\begin{itemize}
\item Setting up Git:
\begin{bash}
git config --global user.name "Suuuuuuuus"
git config --global user.email "bjzzp01@163.com"
git config --global core.autocrlf false
git config --global core.editor "code --wait"
git config --global init.defaultBranch main
\end{bash}
One can check settings by
\begin{bash}
git config --list
\end{bash}
\item Create a repository:
\begin{bash}
git init
\end{bash}
Change the default branch to be called \code{main}:
\begin{bash}
git checkout -b main
\end{bash}
Use the following code to show the status of our project:
\begin{bash}
git status
\end{bash}
\item To tell Git to track a file:
\begin{bash}
git add mars.txt
\end{bash}
as if telling Git to gather stuff before a commit. It is meaningless to \code{add} an empty directory to Git, but if there are uncommited files in the directory, one can \code{add} them all\footnote{One can also use \code{git add -A} or \code{git add .} to add all files in the repository.} by
\begin{bash}
git add spaceships/
\end{bash}
After that, we can let Git commit:
\begin{bash}
git commit -m "Notes"
\end{bash}
with some notes. Without the \code{-m} command, Git will launch some text editor to allow us to write a longer message. This commands is as if taking a snapshot for all stuff that has been \code{add} by Git. Alternatively, one may run
\begin{bash}
git commit -a
\end{bash}
yet it's not recommended. List all commits made to a repository in reverse chronological order:
\begin{bash}
git log
\end{bash}
\begin{itemize}
	\item Note that if the \code{log} is too long, Git might switch to a pager program where one can play with like a manual. To limit the number of commits one makes, simply use
\begin{bash}
git log -1
\end{bash}
	\item Reduce the quantity of information using
\begin{bash}
git log --oneline
\end{bash}
	\item To display the commit history as a text-based graph:
\begin{bash}
git log --graph
\end{bash}
	\item One can combine \code{diff} and \code{log} by
\begin{bash}
git log --patch mars.txt
\end{bash}
It is also possible to specify a specific commit by \code{HEAD~3}.
\end{itemize}
\item Review our changes from the last commit:
\begin{bash}
git diff
\end{bash}
\begin{itemize}
	\item One can check if there is any difference between the previous commit and stuff in the staging area:
\begin{bash}
git diff --staged
\end{bash}
	\item One can review differences between the current file and previous commits by
\begin{bash}
git diff HEAD~3 mars.txt
\end{bash}
where \code{\~3} refers to the third-to-last commit.
	\item One can also specify the ID associated with each commit by \code{6c498cdd333403404ad483dd94330b7a0d641999} or \code{6c498cd}:
\begin{bash}
git diff 6c498cd mars.txt
\end{bash}
\end{itemize}
\item To show what changes we made at an older commit as well as the commit message, we use:
\begin{bash}
git show HEAD~3 mars.txt
\end{bash}
\item The \code{checkout} command allows us to go back to previous commits:
\begin{itemize}
	\item One can go back by one step via:
\begin{bash}	
git checkout HEAD mars.txt
\end{bash}
	\item To go back further:
\begin{bash}
git checkout 6c498cd mars.txt
\end{bash}
	\item If one forgets to put the target file behind \code{checkout}, it might find itself in the "detached HEAD" state. This can be reverted by
\begin{bash}
git checkout main
\end{bash}
	\item Note that \code{git checkout} can also be used to get rid of the staged but not yet committed changes.
\end{itemize}
\item We can keep track of files that we want Git to ignore by creating a file called \code{.gitignore} and put all filenames in. We still need to \code{add} and \code{commit} this file. Note that if any of the files in there were already being tracked, Git would continue to track them.
\begin{itemize}
	\item If one accidentally adds one of the ignored file
\begin{bash}
git add a.dat
\end{bash}
An error message will pop up. However, one can still add it by
\begin{bash}
git add -f a.dat
\end{bash}
	\item We can also always see the status of ignored files if we want:
\begin{bash}
git status  --ignored
\end{bash}
	\item We can use the \code{!} exclamation point operator to except a file from \code{.gitignore}:
\begin{bash}
!final.dat #except final.data
\end{bash}
	\item One can ignore all \code{.dat} files, no matter which subdirectories they are in by putting the following command in \code{.gitignore}:
\begin{bash}
**/*.dat
\end{bash}
\end{itemize}
\item To connect the local and the remote repository, first we need to create a repository on GitHub that has exactly the same name as our local ones. We then copy the SSH link and type:
\begin{bash}
git remote add origin git@github.com:Suuuuuuuus/planets.git
\end{bash}
locally. Whether this is properly done can be checked by:
\begin{bash}
git remote -v
\end{bash}
Then, we need to setup SSH on our local PC. Simply follow instructions on this website\footnote{\url{https://swcarpentry.github.io/git-novice/07-github/index.html}.}. Stuff below is only for record purpose:
\begin{bash}
ssh-keygen -t ed25519 -C "bjzzp01@163.com"
Enter
kotori0803
kotori0803
ssh -T git@github.com
cat ~/.ssh/id_ed25519.pub
\end{bash}
Then, we paste the key into GitHub.
\item To push the changes from our local repository to the repository on GitHub:
\begin{bash}
git push origin main
\end{bash}
\item To pull the changes from GitHub to our local repository:
\begin{bash}
git pull origin main
\end{bash}
One can force two repositories to merge with \code{--allow-unrelated-histories}.
\item If one has accidentally deleted some files, one may use
\begin{bash}
git ls-files --deleted
\end{bash}
to show all deleted files, and then
\begin{bash}
git checkout .
\end{bash}
to retain the target files. Alternatively, it's possible to reset the local branch to what's at remote:
\begin{bash}
git reset --hard origin/main
\end{bash}
\item The \code{git remote} family of commands is used to set up and alter the remotes associated with a repository\footnote{A \textcolor{magenta}{remote} is a copy of the repository that is hosted somewhere else, that we can push to and pull from.}.
\begin{itemize}
	\item \code{git remote -v} lists all the remotes that are configured.
	\item \code{git remote add [name] [url]} is used to add a new remote.
	\item \code{git remote remove [name]} removes a remote. Note that it doesn't affect the remote repository at all - it just removes the link to it from the local repo.
	\item \code{git remote set-url [name] [newurl]} changes the URL that is associated with the remote.
	\item \code{git remote rename [oldname] [newname]} changes the local alias by which a remote is known.
\end{itemize}
\item To get the remote changes into the local repository but without merging them, one can run
\begin{bash}
git fetch origin main
\end{bash}
Then by running
\begin{bash}
git diff main origin/main
\end{bash}
one can see the changes output in the terminal.
\item Miscellaneous:
\begin{itemize}
\item Use the command
\begin{bash}
git rebase --abort
\end{bash}
to recover the lost files\footnote{I have no idea what happened before but I accidentally lost many files (quite randomly). After running this command I have everything back.}.
\item Here is a link\footnote{\url{https://swcarpentry.github.io/git-novice/14-supplemental-rstudio/index.html}.} for using Git in RStudio.
\item When trying to upload files larger than 100MB, one can refer to this link\footnote{\url{https://blog.csdn.net/qq\_42196916/article/details/105812410?utm\_medium=distribute.pc\_relevant.none-task-blog-2~default~baidujs\_baidulandingword~default-5-105812410-blog-105779097.pc\_relevant\_aa2&spm=1001.2101.3001.4242.4&utm\_relevant\_index=8}.} for solution.
\end{itemize}
\end{itemize}
\end{document}
